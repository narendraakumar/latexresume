%%%%%%%%%%%%%%%%%%%%%%%%%%%%%%%%%%%%%%%%%
% Twenty Seconds Resume/CV
% LaTeX Template
% Version 1.0 (14/7/16)
%
% Original author:
% Carmine Spagnuolo (cspagnuolo@unisa.it) with major modifications by 
% Vel (vel@LaTeXTemplates.com) and Harsh (harsh.gadgil@gmail.com)
%
% License:
% The MIT License (see included LICENSE file)
%
%%%%%%%%%%%%%%%%%%%%%%%%%%%%%%%%%%%%%%%%%

%----------------------------------------------------------------------------------------
%	PACKAGES AND OTHER DOCUMENT CONFIGURATIONS
%----------------------------------------------------------------------------------------

\documentclass[letterpaper]{twentysecondcv} % a4paper for A4

% Command for printing skill overview bubbles
\newcommand\skills{ 
~
	\smartdiagram[bubble diagram]{
        \textbf{Data}\\\textbf{Engineering},
        \textbf{Image}\\\textbf{Processing},
        \textbf{~~~~~~~~OOP~~~~~~~~~},
        \textbf{~~~~~~Machine~~~~~~}\\\textbf{~~Learning~~},
        \textbf{~~~~~Opencv~~~~~}
    }
}

% Programming skill bars
\programming{{C $\textbullet$ C++ /1.4 }, { Computer Vision / 4}, { Python / 5}}

% Projects text
\education{
\textbf{MTech., Aerospace Engineering} (GPA: 8.30) \\
Specialization: Aerospace Structure \\
Indian Institute of Technology  \\
2015 - 2017 | Kharagpur, India

\textbf{BEng., Aeronautical Engineering} (First Class) \\
Aeronautical Society Of India \\
2010 - 2015 | Delhi
}

%----------------------------------------------------------------------------------------
%	 PERSONAL INFORMATION
%----------------------------------------------------------------------------------------
% If you don't need one or more of the below, just remove the content leaving the command, e.g. \cvnumberphone{}

\cvname{Narendra Kumar} % Your name
\cvjobtitle{Software Engineer(ML) } % Job
% title/career

\cvlinkedin{/in/naren-ar}
\cvgithub{}
\cvnumberphone{+91-9958508703} % Phone number
\cvsite{https://narendraakumar
.github.io/curriculum-vitae/} % Personal website
\cvmail{nareaero@gmail.com} % Email address

%----------------------------------------------------------------------------------------

\begin{document}

\makeprofile % Print the sidebar
 
%----------------------------------------------------------------------------------------
%	 EXPERIENCE
%----------------------------------------------------------------------------------------

\section{Experience}

\begin{twenty} % Environment for a list with descriptions
\twentyitem
    	{Jan 2018 -}
		{Present}
		{\textbf{Python Developer Build AI Solution for Tech Startup Exponential Machines}}
        {\href{http://www.xpms.io/}{}}
        {}
        {\begin{itemize}
		\item \textbf{Projects :} Extraction of the different entity from scanned documents.
%	\item Project involved developing microservice.
	\item Worked on Platform that involves microservice architecture.
	\item Working knowledge on DataBases like \textbf{Mongo, Cassandra, Hbase}
        \item Focused on data extraction from documents for diffrent machine learning algorithms.
		\item  Experience in applying machine learning algorithms in computer vision, NLP.
%		\item Knowledge of data science technologies such as Pandas, Scipy, Numpy, matplotlib, etc.
		\item Computer Vision knowledge - Extensively worked on image processing using opencv framework.
		\item \textbf{Project :} Working in NLP team under Product Development Group. Our team takes care of finding insights in data obtained from
		different kind of documents like Medical Charts. Worked on following problems.
		\item Single-handedly developing product which finds ICD 10 code from medical reports. Doing classification modelling and
		entity extraction for the project using Spacy and Ctakes. 
		\item Developed tool for summarization of text. This works nicely on news articles and research papers.
        

        \end{itemize}}
        \\
% 	\twentyitem
%     	{Sep 2015 -}
% 		{May 2016}
%         {Co-founder \& Full Stack Developer}
%         {\href{http://www.localxchange.ca/}{LocalXChange Inc.}}
%         {}
%         {
%         {\begin{itemize}
%         \item In a team of 2, raised \$8,000 in funding from The Hub incubator at the University of Guelph, to develop a hyperlocal content platform, aimed at delivering local news and events to local users in realtime
%         \item In a team of 3, built hybrid mobile \& web apps with Ionic, Angular.js and MongoDB, surpassing 1,000 users within a month since launch
%         \item Met with the Mayor of Guelph and University of Guelph officials to discuss how the app can help boost Guelph Tourism 
%     \end{itemize}}
%         }
%     \\   
    \twentyitem
   		{Sep 2016 -}
		{May 2017}
        {Graduate Teaching Assistant}
        {\href{http://www.iitkgp.ac.in}{IIT Kgp}}
        {}
        {
        {\begin{itemize}
        \item Teaching Assistant During M-Tech program.
    \end{itemize}}
        }
%      \\
%      \twentyitem
%   		{Dec 2013 -}
% 		{Apr 2015}
%         {Test Automation Engineer}
%         {\href{http://www.synechron.com/}{Synechron}}
%         {}
%         {
%         \begin{itemize}
%         \item Primarily developed test automation libraries using Java and C\#, scripts and CI / CD pipelines
%         \item Lead development of a Keyword Driven \& Behavior Driven test framework for \href{https://www.microsoft.com/en-ca/dynamics/crm.aspx}{Microsoft Dynamics CRM}. Earned monetary award \& client appreciation.
%         \item Demonstrated that rewriting an in-house test framework for \href{https://www.microsoft.com/en-ca/dynamics/erp-ax-overview.aspx}{Microsoft Dynamics AX}, using an open source library (White) instead of a proprietary one (Coded UI), would help the team save \$4K annually by downgrading Microsoft Visual Studio
%     \end{itemize}
%     	}
        
	%\twentyitem{<dates>}{<title>}{<location>}{<description>}
\end{twenty}

%----------------------------------------------------------------------------------------
%	 RESEARCH
%----------------------------------------------------------------------------------------
\section{Expertise}
\begin{twenty}
	\twentyitem
    	{}
		{}
        {}
        {\href{}{}}
        {}
        {
        \vspace{-6 mm}

        {\begin{itemize}
		\item Image Processing using Python.
		\item Text extraction from images.
		\item Text Mining, Text Classification, Document Classification.
	\item Python Package uses \textbf{OpenCV, Numpy, Pandas, nltk, tesseract, ocropus}
		\item \textbf{Operating system Ubuntu and IDE Pycharm}
        \item \textbf{Tools}: Python, scikit-learn, pandas, MongoDB, RabbitMQ, redis, JIRA
        \vspace{2mm}
		\end{itemize}}
        }
\end{twenty}
\section{Course Certification}

\begin{twenty} % Environment for a list with descriptions
	
%	\twentyitem
%    	{Dec - 2016  }
%    	{}
%        {Web data analytics using Python, IIT Kharagpur}
%        {\href{http://www.gian.iitkgp.ac.in//}{Short Term Course}}
%        {}
%        {
%        % \vspace{-2mm}
%        {\begin{itemize}
%        \item First module includes text extraction process, pre-processing and
%        text processing and sentiment analysis of web log file.
%       \item Second module Web log analysis using Python that has data processing,data collection, data cleaning, and modeling of user navigation
%       behavior.
%        \end{itemize}}
%        \hspace{3 mm}
%        }
%
    \twentyitem
   		{Dec - 2016 }
   		{}
        {Machine Learning}
        {\href{https://www.coursera.org/accomplishments//}{Online certification from Stanford University on Coursera}}
        {}
        {
         \vspace{-2mm}
        {\begin{itemize}
        \item Analyzing dataset to identify kind of patterns based on their behavior. Applying machine learning methods, principal component analysis, logistic regression on the large dataset to build the predictive model. Python uses extensively for analysis and dimensional visualization.
        \item This course contains linear, logistic regression, classification problem, neural network, etc.
    \end{itemize}}
        }

	%\twentyitem{<dates>}{<title>}{<location>}{<description>}
\end{twenty}
\vspace{3 mm}
\section{Research}
\begin{twenty}
\twentyitem
    	{2015 - 2017}
    	
        {Master's Project}
        {\href{http://http://iitkgp.ac.in//}{IIT Kharagpur}}
        {}
        {
        	\vspace{-3mm}
        {\begin{itemize}
        
        
        \vspace{1.5 mm}
		\item Detection of Delamination in the composite beam using ultrasonic wave propagation technique. Modeling of the beam is done using FEM and code written in MATLAB.
		\item Detection of crack in the beam using ultrasonic wave propagation method. Modeling of the beam is done using FEM and code were written in MATLAB.
\end{itemize}}
        }
\end{twenty}


\end{document} 
